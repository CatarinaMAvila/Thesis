%!TEX root = ../template.tex
%%%%%%%%%%%%%%%%%%%%%%%%%%%%%%%%%%%%%%%%%%%%%%%%%%%%%%%%%%%%%%%%%%%%
%% chapter3.tex
%% NOVA thesis document file
%%
%% Chapter with a short latex tutorial and examples
%%%%%%%%%%%%%%%%%%%%%%%%%%%%%%%%%%%%%%%%%%%%%%%%%%%%%%%%%%%%%%%%%%%%

\typeout{NT FILE chapter3.tex}%

\makeatletter
\newcommand{\ntifpkgloaded}{%
  \@ifpackageloaded%
}
\makeatother


\chapter{Work Plan}
\label{cha:workplan}
As it was proposed in chapter~\ref{cha:introduction} and the title, the main purpose of this thesis is the design, simulation and optimization of \glspl{MPA} for Wi-Fi 7.
The development approach includes theorectical research, preliminary designs, simulation in MATLAB\textsuperscript{TM} and fabrication of the designed antenna for validation of the results.
An initial analysis was needed to better understand the standards, techniques and multiband approaches for \gls{MPA}, this can be seen in chapter~\ref{cha:stateoftheart}. 
Only after that, the design, simulation and optimization part of this thesis actually start. 
In the following section a task breakdown of the development approach is presented.
A timeline of the tasks will also be presented as well as the preliminary work and the expected deliverables.

\section{Tasks}
Following that explanation, the tasks proposed for the development of this thesis are as follows:
\begin{description}
  \item[Task 1 - Background Analysis and State-of-the-art:] In this first task, research is made to better understand Wi-Fi 7 and \glspl{MPA}. Different \gls{MPA} methods were studied, and a literature review is conducted including some multiband methods for the \gls{MPA}.
  \item[Task 2 - Specifications Definition and Preliminary \gls{MPA} Designs:] In the second task, the antenna's specification will be defined based on Wi-Fi 7 standards and an analysis of the optimal geometry/substrate/feeding methods will be conducted. An antenna will be designed using MATLAB\textsuperscript{TM} scripting.
  \item[Task 3 - Simulation and Optimization:] In the third task, the main figures of merit, including radiation patterns, S-parameters and radiation efficiency will be simulated. Then a multi-objective optimization that includes area, bandwidths, antenna gain and performance stability will be conducted using MATLAB\textsuperscript{TM}. In this phase some \gls{AI} optimization techniques will also be cosidered.
  \item[Task 4 - Fabrication and Measurements:] In the fourth task, final considerations and insights will be thought off. The project files will be generated using MATLAB\textsuperscript{TM} (Gerber) and the antenna PBC will be manufactured. After that, the performance metrics will be measured and a final discussion and comparison between theory, simulation and experimental data will be made.
  \item[Task 5 - Thesis writting:] The fifth task, will run in parallel with the other tasks, as it is important to do it continuously so that no detail from other tasks is forgotten.
\end{description}

\section{Timeline}
The time allocated for each task will be as follows:

\begin{figure}[h]
    \centering
    \includegraphics[width=\textwidth]{5-Figures/Thesis timetable.pdf}
    \caption{Timeline for the tasks in a Gantt chart (October 2025 - September 2026)}
    \label{fig:thesis-timeline}
\end{figure}

\section{Preliminary Work and Expected Deliverables} 
The preliminary work that was done for this thesis was only theorectical, as it was essencial to learn about an antenna's main performance metrics, some of its geometries, different feeding, sizing and bandwidth enhancement methods.
It was also important to learn the basics of a rectangular \gls{MPA} analysis and on the other side of the spectrum, it was essencial to learn different techniques for multiband \glspl{MPA} and some \gls{AI} optimization methods.
For the simulation part of this thesis, MATLAB\textsuperscript{TM} was the chosen tool, as it has a vast library for antenna design and simulation including some pre-made examples.
\vspace{1cm}

The expected deliverables for this thesis are:
\begin{itemize}
  \item the final design parameters of the optimized \gls{MPA} for Wi-Fi 7;
  \item the final MATLAB\textsuperscript{TM} script;
  \item the Gerber files generated from MATLAB\textsuperscript{TM}/CST;
  \item the manufactured \gls{MPA};
  \item the performance metrics measurements of the final \gls{MPA};
  \item the comparision results between theory, simulation and experimental data.
\end{itemize}

% \printbibliography[heading=subbibliography, segment=\therefsegment, title={\bibname\ for chapter~\thechapter}]

