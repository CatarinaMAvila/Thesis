%!TEX root = ../template.tex
%%%%%%%%%%%%%%%%%%%%%%%%%%%%%%%%%%%%%%%%%%%%%%%%%%%%%%%%%%%%%%%%%%%
%% chapter1.tex
%% NOVA thesis document file
%%
%% Chapter with introduction
%%%%%%%%%%%%%%%%%%%%%%%%%%%%%%%%%%%%%%%%%%%%%%%%%%%%%%%%%%%%%%%%%%%

\typeout{NT FILE chapter1.tex}%

\chapter{Introduction}
%\label{cha:introduction}

Wireless communication systems are technologies that transfer information through the air without any physical wires~\cite{wireless_review_2020}. 
This way of communicating can be achieved through the use of a \gls{WLAN}~\cite{overview_WLAN}. 
A \gls{LAN} is a network of computers linked together in a limited geographic area whilst a \gls{WLAN} is the equivalent without any wire constraints. 
A \gls{WLAN} is built on a cellular architecture that divides a system into cells with each being administered by a base station~\cite{detailed_study_WLAN}.  
The benefits of using a \gls{WLAN} are the mobility, flexibility, ease of management, easy of maintenance and lowered cost~\cite{overview_WLAN}. 
There are multiple standards to a \gls{WLAN}, also called Wi-Fi standards, some of them being 802.11n (Wi-Fi 4), 802.11ac (Wi-Fi 5), 802.11ax (Wi-Fi 6 and Wi-Fi 6E) and 802.11be (Wi-Fi 7), with these last two being the latest. 
802.11ax offered greater throughput, better energy efficiency and enhanced performance in crowded environments~\cite{evolution_WIFI_standards}, but with the increased demand for faster speeds and enhanced reliability in wireless communications~\cite{evolution_WIFI_standards}, the Wi-Fi 7 is the most recent solution developed. 
The main improvements in Wi-Fi 7 that make it stand out from its predecessor are a channel bandwidth of up to 320 MHz that allows a more efficient and high-speed data transfer. 
The use of 4096-QAM modulation that lets the data that is transferred to be packed with more information in each radio wave and the introduction of the \gls{MLO}, that grants users the ability to connect to multiple frequency bands, like 2.4 GHz, 5 GHz, 6 GHz at the same time, giving the user a stable connection with reduced delay\cite{revista_foz_351}. 

\section{Problem Statement and Motivation}
One of the most common antenna technologies for Wi-Fi 7 and the one this thesis will be focusing on is the \gls{MPA}, also called patch antennas. 
These are a type of antenna that consists of a metallic patch on a dielectric substrate with a ground plane on the other side. 
According to~\cite{microstrip_review_2019} the \gls{MPA} was conceptualized in 1953 but only became practical by the 1970’s when low loss substrate materials became more available. 
At that time patch antennas were being mostly used in spaceborne applications~\cite{balanis2016antenna}, in current times the \glspl{MPA} are still used for aerospace applications and low-profile communications~\cite{garg2001microstrip}. 

\noindent \glspl{MPA} are very versatile in their design and can be made to operate at multiple frequencies, as well as adapted to improve their shortcommings, such as their narrow bandwidth and low gain~\cite{microstrip_review_2019}.
Therefore, motivation for this thesis is to answer the following question:
\begin{quote}
\textit{What are the best techniques to create/optimize an antenna of the type microstrip patch that efficiently operates in three different Wi-Fi 7 (802.11be) bands?}
\end{quote}
This problem is relevant because in the current world, there is a growing need for high-speed wireless communication devices that can also handle the volume of data of today's applications~\cite{revista_foz_351}, and
\glspl{MPA} allow for these objectives to be acheived in compact devices due to their advantages\cite{microstrip_review_2019}.

\section{Objectives}
To properly find an answer to the proposed question, we must first define some objectives:
\begin{itemize}
    \item Design of a multiband \gls{MPA};
    \item Simulation of the designed antenna using MATLAB;
    \item Antenna optimization;
    \item Fabrication and measurement of the antenna's performance.
\end{itemize}

\section{Document Structure}
This document is devided into four chapters. The first chapter is this one, and is where the context, problem statement and objectives are presented.
The second chapter is a literature review on \glspl{MPA}, where its fundamentals are explained as well as the implementation techniques for multiband \glspl{MPA} are shown.
The third chapter presents this thesis' work plan, a breakdown of the tasks and respective timeline as well as work done and what is expected to be delivered.
Finally, the fourth and last chapter presents some conclusions of the work done and some future work to be made.