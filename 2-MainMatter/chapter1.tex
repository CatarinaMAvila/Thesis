%!TEX root = ../template.tex
%%%%%%%%%%%%%%%%%%%%%%%%%%%%%%%%%%%%%%%%%%%%%%%%%%%%%%%%%%%%%%%%%%%
%% chapter1.tex
%% NOVA thesis document file
%%
%% Chapter with introduction
%%%%%%%%%%%%%%%%%%%%%%%%%%%%%%%%%%%%%%%%%%%%%%%%%%%%%%%%%%%%%%%%%%%

\typeout{NT FILE chapter1.tex}%

\chapter{Introduction}
%\label{cha:introduction}

Wireless communication systems are technologies that transfer information through the air without any physical wires [1]. 
This way of communicating can be achieved through the use of a wireless local area network, also known as WLAN [2]. 
A LAN is a network of computers linked together in a limited geographic area whilst a WLAN is the equivalent without any wire constraints. 
A WLAN is built on a cellular architecture that divides a system into cells with each being administered by a base station [3].  
The benefits of using a WLAN are the mobility, flexibility, ease of management and maintenance and lowered cost [2]. 
There are multiple standards to a WLAN, also called Wi-Fi standards, some of them being 802.11n (Wi-Fi 4), 802.11ac (Wi-Fi 5), 802.11ax (Wi-Fi 6 and Wi-Fi 6E) and 802.11be (Wi-Fi 7), with these last two being the latest. 
802.11ax offered greater throughput, better energy efficiency and enhanced performance in crowded environments [4] but with the increased demand for faster speeds and enhanced reliability in wireless communications [4] the Wi-Fi 7 is the most recent solution developed. 
The main improvements in Wi-Fi 7 that make it stand out from its predecessor are a channel bandwidth of up to 320 MHz that allows a more efficient and high-speed data transfer, the use of 4096-QAM modulation that lets the data that is transferred to be packed with more information in each radio wave and the introduction of the Multi-Link Operation (MLO), that grants users the ability to connect to multiple frequency bands, like 2.4 GHz, 5 GHz, 6 GHz at the same time, giving the user a stable connection with reduced delay[5]. 
One of the most common antenna technologies for Wi-Fi 7 and also the one this thesis will be focusing on is the Microstrip Patch Antenna also called patch antennas or MPAs. 
These are a type of antenna that consists of a metallic patch on a dielectric substrate with a ground plane on the other side. 
According to [6] the microstrip patch antenna was conceptualized in 1953 but only became practical by the 1970’s when low loss substrate materials became more available. 
At that time patch antennas were being mostly used in spaceborne applications [7], in current times the MPAs are still used for aerospace applications and low-profile communications [8]. 

\section{Problem Statement and Motivation}
TO DO
\section{Objectives}
TO DO