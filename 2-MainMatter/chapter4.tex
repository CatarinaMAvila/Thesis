%!TEX root = ../template.tex
%%%%%%%%%%%%%%%%%%%%%%%%%%%%%%%%%%%%%%%%%%%%%%%%%%%%%%%%%%%%%%%%%%%%
%% chapter4.tex
%% NOVA thesis document file
%%
%% Chapter with lots of dummy text
%%%%%%%%%%%%%%%%%%%%%%%%%%%%%%%%%%%%%%%%%%%%%%%%%%%%%%%%%%%%%%%%%%%%

\typeout{NT FILE chapter4.tex}%

\chapter{Conclusions}
%\label{cha:porting_novathesis}
In this final chapter, some conclusions and future work will be presented.
This preparatory report focuses mostly on the theorectical part of the work, where first we understand \glspl{MPA} and their characteristics, and then present possible techniques to answer the proposed question.
In the first part of chapter two, we saw the advantages and limitations of \glspl{MPA}, as well as, their main performance parameters and the analysis of a basic rectangular \gls{MPA}.
Then, in the second part of the same chapter, we saw multiple geometries and techniques for sizing, bandwidth enhancement, feed and methods to acheive dual and tri-band operation, as well as, some use cases of multiband \glspl{MPA}.
Finally in the end of that chapter, we looked at some \gls{AI} optimization techniques for the design of \glspl{MPA}.
In the third chapter of this work, a timeline and work plan was presented, with tasks, which include the objectives defined in chapter one, to find the solution to the problem.

The future work is the major step to answer the proposed question. 
Starting with the definition of the specifications for the \gls{MPA} and its preliminary design. 
After this, the chosen design will be simulated to find the antenna's performance metrics and then optimized.
Once the final design is acheived, the fabrication of the antenna will proceed and real world measurements will be made. 
These measurements will then be compared with those obtained in the simulation to validate the design.